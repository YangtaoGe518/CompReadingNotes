\documentclass[12pt]{article}
\usepackage[utf8]{inputenc}
\usepackage{comment}
\usepackage{listings}
\usepackage{mathtools}

\setlength{\parindent}{0em}
\setlength{\parskip}{0.5em}

\title{Exceptions, Assertions and Logging}
\author{Yangtao Ge}
\date{\today}

\begin{document}
\maketitle
This section will talk about three topics
\begin{itemize}
    \item \textbf{Exception handling}: use some cases to avoid accidental errors
    \item \textbf{Assertions}: run several checks to make sure you program does the right thing
    \item \textbf{Logging}: record problems into files
\end{itemize}

\section{Dealing with Errors}
Basic Requirement:
\begin{itemize}
    \item Return to a safe state and enable the user to execute other commands
    \item Allow user to save all work and terminate the program \underline{gracefully}
\end{itemize}

Possible Errors:
\begin{itemize}
    \item User input errors: \textit{syntatically} wrong
    \item Device errors: Hardware may not be able to do what you want (Power off?)
    \item Physical limitations: Disks can be filled up 
    \item Code errors: using something in a wrong way for existing codes
\end{itemize}

In Java, we use '\underline{throw}' to provide an object \textit{which encapsulates the error information}

\subsection{The Classification of Exception}
Here is the hierarchy of Exception in Java:\newline

Class \textbf{Throwable}:
\begin{itemize}
    \item Error: ...
    \item Exception: 
    \begin{itemize}
        \item IOException:
        \begin{itemize}
            \item read past the end of a file
            \item open a file that doesn't existing
            \item find a \textit{Class} Object for a string that does not denote an existing class 
        \end{itemize}
        \item Runtime Exception: \underline{RuntimeException means it was your fault}
        \begin{itemize}
            \item A bad cast
            \item An out of bounds array access
            \item A null pointer access
        \end{itemize}
    \end{itemize}
\end{itemize}

Here is the classification of Exception:
\begin{itemize}
    \item \textbf{unchecked} Exception:
    \begin{itemize}
        \item class \textbf{Error}
        \item class \textbf{RuntimeException}
    \end{itemize}
    \item \textbf{checked} Exception: \textit{Others}
\end{itemize}

\subsection{Declaring Checked Exception}
Aim of throwing exceptions:
\begin{itemize}
    \item tell the Java Compiler \underline{what values it can return}
    \item tell the Java Compiler \underline{what can go wrong}
\end{itemize}

We can throw an exception from a method or a constructor(Class). 

For a constructor, when we intialize an object, either it will produce an object correctly, 
or it will thrown an Exception object. This is the same as methods.

For a method, we \textbf{do not} need to throw every possible exceptions. Here are 4 situations where exception
will be thrown:
\begin{itemize}
    \item Call a method that throws a \textbf{checked} exception
    \item Detect an \textbf{error} or a \textbf{checked} exception with the `\textit{throw}' statement
    \item Make a programming error which rise to an \textbf{unchecked} exception
    \item An internal error occurs in the \underline{virtual machine} or \underline{runtime library}.
\end{itemize}

In summary:
\begin{itemize}
    \item a method must declare all the \underline{checked} exception that it might throw.
    \item \underline{unchecked} exceptions are either beyond your control or result from conditions that
    we should not allow in the first place 
\end{itemize}

\subsection{How to Throw an Exception}
An example for how to throw an exception:
\begin{lstlisting}[language=Java]
    String readData(Scanner in) throws EOFException{
        ...
        while(...){
            if(!in.hasNext()){
                if(n<len) throw new EOFException();
            }
        }
        return s;
    }
\end{lstlisting}

and we could define the exception message if we wish
\begin{lstlisting}[language=Java]
String gripe = ``content-length: "+ len +``, Received: " + n;
throw new EOFException(gripe);
\end{lstlisting}

\subsection{Creating Exception Classes}
we can create our own exception classes:
\begin{lstlisting}[language=Java]
    class FileFormatException extends IOException{
        public FileFormatException(){}
        public FileFormatException(){
            super(gripe);
        }
    }
\end{lstlisting}

\section{Catching Exceptions}
Catching needs more \underline{plans} than throwing
\subsection{Catching an Exception}


\end{document}