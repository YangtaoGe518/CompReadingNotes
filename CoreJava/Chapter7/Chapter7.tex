\documentclass[12pt]{article}
\usepackage[utf8]{inputenc}
\usepackage{comment}
\usepackage{listings}
\usepackage{mathtools}

\setlength{\parindent}{0em}
\setlength{\parskip}{0.5em}

\title{Exceptions, Assertions and Logging}
\author{Yangtao Ge}
\date{\today}

\begin{document}
\maketitle
This section will talk about three topics
\begin{itemize}
    \item \textbf{Exception handling}: use some cases to avoid accidental errors
    \item \textbf{Assertions}: run several checks to make sure you program does the right thing
    \item \textbf{Logging}: record problems into files
\end{itemize}

\section{Dealing with Errors}
Basic Requirement:
\begin{itemize}
    \item Return to a safe state and enable the user to execute other commands
    \item Allow user to save all work and terminate the program \underline{gracefully}
\end{itemize}

Possible Errors:
\begin{itemize}
    \item User input errors: \textit{syntatically} wrong
    \item Device errors: Hardware may not be able to do what you want (Power off?)
    \item Physical limitations: Disks can be filled up 
    \item Code errors: using something in a wrong way for existing codes
\end{itemize}

In Java, we use '\underline{throw}' to provide an object \textit{which encapsulates the error information}

\end{document}