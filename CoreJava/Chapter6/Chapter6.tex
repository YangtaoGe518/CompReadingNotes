\documentclass[12pt]{article}
\usepackage[utf8]{inputenc}
\usepackage{comment}
\usepackage{listings}
\usepackage{mathtools}

\setlength{\parindent}{0em}
\setlength{\parskip}{0.5em}

\title{Interfaces, Lambda Expressions, and Inner Classes}
\author{Yangtao Ge}
\date{\today}

\begin{document}
\maketitle

\section{Interfaces}
\subsection{The Interface Concept}
\underline{Definition:} \emph{\textbf{Interface}} is not a class but a set of \emph{requirements} for the classes
that we want to conform to the interface. \newline
e.g.
\begin{lstlisting}[language=Java]
    public interface Comparable<T>{
        int compareTo(T other); // para has type T
    }
\end{lstlisting}

Some Notice to Interface:
\begin{itemize}
    \item all methods of an interface are automatically \underline{\textit{public}} -- we don't add \emph{public} in the signiture
    \item interfaces can define constants
    \item interfaces cannot have instance fields
    \item method are never implemented in interface (we could now, but it's \textbf{BAD})
\end{itemize}

To make a class implement an interface:
\begin{enumerate}
    \item declare that your class intends to \emph{implement} the given interface
    \item supply definitions for \underline{all} methods in the interface
\end{enumerate}
e.g.
\begin{lstlisting}[language=Java]
    class Employee implements Comparable<Employee>{
        public int compareTo(Employee other){
            return Double.compare(salary, other.salary);
        }
        ... 
    }
\end{lstlisting}
N.B. Try to use \emph{Generic type}, less using `type cast'.

Regarding to \textit{compareTo} method:
\begin{itemize}
    \item how to compare:
    \begin{itemize}
        \item substraction: if we know the maximum bounday is less than `maximum of Integer'.
        \item compareTo: don't care
    \end{itemize}
    \item inheritance Problems: (solve like \emph{equal()} method in Chapter 5)
    \begin{itemize}
        \item different notations of comparison: add a same class test
        \begin{lstlisting}[language=Java]
        if(getClass() != other.getClass){
            throw new ClassCastException;
        }    
        \end{lstlisting}
        \item common algorithm: provide a single \emph{compareTo} method, and declare it as \emph{final}
    \end{itemize}
\end{itemize}

\subsection{Properties of Interfaces}
Some properties of Interface:
\begin{itemize}
    \item Interfaces are not classes -- can't do `\emph{x = new Comparable(...);}'
    \item we can declare \underline{interface variables}
    \begin{itemize}
        \item can do `\emph{Comparable x;}'
        \item can do `\emph{x = new Employee(...);}' (since `Employee' implements Comparable)
    \end{itemize}
    \item we can check whether an object implements an interface by `\emph{instanceof}' keyword
    \item we can extend Interfaces
    \begin{lstlisting}[language=Java]
    public interface Moveable{
        void move (double x, double y);
    }
    public interface Powered extends Moveable{
        double milesPerGallon();
    }
    \end{lstlisting}
    \item we can add \emph{constants} in the interface. This mathod is automatically `\textbf{public static final}'
    \begin{lstlisting}[language=Java]
    public interface Powered extends Moveable{
        double milesPerGallon();
        // public static final constant
        double SPEED_LIMIT = 95; 
    }
    \end{lstlisting}
    \item classes can implement \emph{multiple} interfaces -- we can do `\emph{class Employee implements Person, Comparable}'
    (but one class can only have one superclass)
\end{itemize}

\subsection{Interfaces and Abstract Classes}
\underline{Key:} A class can only extend a single class, but can implement several interfaces.

We can think it as:
\begin{itemize}
    \item abstract classes: tends to stress what it is (inheritance -- `is-a relationship')
    \item interface: tends to illustrate what can it do (properties).
\end{itemize}
N.B. Remember it by -- \textbf{things can only belong to one class, but it can have several properties.}

\subsection{Static and Private Methods}
We can add '\emph{static} method' since Java 8, and `\emph{private} method' since Java 9.
This is not very useful.

\textit{Ref: p.306}

\subsection{Default Methods}
some useful situation for \emph{default} modifier:
\begin{itemize}
    \item implement `iterator': providing an exception
    \begin{lstlisting}[language=Java]
    public interface Iterator<E>{
        boolean hasNext();
        E next();
        default void remove(){
            throw new UnsupportedOperationException(``remove'')
        }
    }
    \end{lstlisting}
    \item implement `collection': call other methods
    \begin{lstlisting}[language=Java]
    public interface Collection{
        int size(); // an abstract method
        default boolean isEmpty(){
            return size() == 0;
        }
    }
    \end{lstlisting}
    \item \emph{interface evolution} for adding class in the future.
\end{itemize} 

\subsection{Resolving Default Method Conflicts}
Two basic rules for resolving default method conflicts:
\begin{itemize}
    \item \textbf{Class win rule}: if a superclass provides a conrete method, 
    default methods with the same name and parameter types are ignored
    \begin{lstlisting}[language=Java]
    class Student extends Person implements Named{
        ... // use getName in Person class only
    }
    \end{lstlisting}

    \item \textbf{Interfaces clash rule}: if an interface provide default method, 
    another interface contains a method with the same name and parameter (types default or not),
    then you must resolve the conflict by overriding the method 
    \begin{lstlisting}[language=Java]
    class Student implements Person, Named{
        public String getName(){
            //choose to use getName in Person
            return Person.super.getName();
        }
    }
    \end{lstlisting}
\end{itemize}

\subsection{Interfaces and Callbacks}
\underline{Definition:} \emph{callback} pattern means when you specify the action that should 
happen whenever a particular event happens. e.g. \emph{ActionListener} in java swing

Usually, we will \underline{predefine} how a method work of a method, then call it whenever we want it. 

\textit{Ref: pp.310 - 312 \& COMP 0004 Java Coursework Part2}

\subsection{The \emph{Comparator} Interface}
What is is like:
\begin{lstlisting}[language=Java]
    public interface comparator<T>{
        int compare(T first, T second);
    }
\end{lstlisting}

Using user-defined comparator:
\begin{lstlisting}[language=Java]
public class LengthComparator implements Comparator<String>{
    public int compare(String first, String second){
        return first.length() - second.length();
    }
}

String friends = {``Peter'', ``Paul'', ``Mary''};
Array.sort(friends, new LengthComparator());
\end{lstlisting}

\emph{Ref: pp.323 - 314 \& p.322}

\subsection{Object Cloning}
Difference between `copy' and `clone':
\begin{itemize}
    \item copy (=): make a copy of variable holding an object reference -- change to either variable also affects the other
    \newline i.e. original $\rightarrow$ Employee $\leftarrow$ copy
    \item clone (clone()): identical to original but whose state can diverge over time
    \newline i.e. original $\rightarrow$ Employee1; cloned $\rightarrow$ Employee2
\end{itemize}

clone method is `\emph{protected}' so that it can only clone itself. i.e. Employee's clone can clone Employee only

Two types of Clone:
\begin{itemize}
    \item \textbf{Shallow copy}: (default) just copy the object only, don't care what are inside it. -- only valid for immutable objects
\begin{lstlisting}[language=Java]
class Employee implements Cloneable{
    //public access, changfe return type
    public Employee clone() throws CloneNotSupportedException{
        return (Employee) super.clone();
    }
 }
\end{lstlisting}

    \item \textbf{Deep copy}: (redefined) clone the instance fields piece by piece -- for mutable objects
\begin{lstlisting}[language=Java]
class Employee implements Cloneable{
    ...
    public Employee clone() throws CloneNotSupportedException{
        //call Object.clone
        Employee cloned = (Employee)super.clone
        //clone mutable fields
        clone.hireDay = (Date)hireDay.clone
    }
}
\end{lstlisting}
\end{itemize}
i.e. \emph{super} means \underline{Cloneable interface}

Condition to use clone:
\begin{itemize}
    \item it can be cloned:
    \begin{itemize}
        \item default clone is good enough (shallow clone)
        \item default clone can be patched up by calling \emph{clone} on mutable sub-objects
    \end{itemize}
    \item it cannot be cloned
\end{itemize}

When it is cloneable:
\begin{enumerate}
    \item implement the \emph{Cloneable} interface
    \item redefine the \emph{clone} method with the \emph{public} accessor
\end{enumerate}

\section{Lambda Expression}
\subsection{Why Lambdas?}
\underline{Definition:} \emph{Lambda expression} is a block of code that you can pass around so it can be
executed \textbf{later, once or multiple times}

Java will not pass the actual code block around. Lambda expression provides a tag (API) to access them.

\subsection{The Syntax of Lambda Expression}
Three types of lambda expression:
\begin{itemize}
    \item simple return:
    \begin{lstlisting}[language=Java]
    (String first, String second) 
        -> first.length()-second.length()
    \end{lstlisting}
    \item code block:
    \begin{lstlisting}[language=Java]
    (String first, String second) -> {
        if (first.length() < second.length()) return -1;
        else if (first.length() < second.length()) return -1;
        else return 0;
    }
    \end{lstlisting}
    \item no parameters:
    \begin{lstlisting}[language=Java]
    () -> {
        for (int i = 100; i>= 0; i--) System.out.println(i);
    }
    \end{lstlisting}
\end{itemize}

i.e. We never specify the `\emph{Return type}', it is inferred from the context.

\subsection{Function Interface}
\underline{Definition:} \emph{functional interface} is an \textbf{interface} that we can supply a lambda expression
whenever an object of an interface with a \underline{single abstract method} is expected.
(i.e. it is implement an interface by `lambda expression')

Some applications of functional interface:
\begin{itemize}
    \item \textbf{Predicate}:
    \begin{lstlisting}[language=Java]
    public interface Predicate<T>{
        boolean test(T t);
    }
    \end{lstlisting}
    The `ArrayList' class has a \emph{removeIf} method whose para is a `Predicate'
    \begin{lstlisting}[language=Java]
    list.removeIf(e -> e == null)
    \end{lstlisting}
    \item \textbf{Supplier}: used for lazy evalutation
    \begin{lstlisting}  [language=Java]
    public interface Supplier<T>{
        T get();
    }
    \end{lstlisting}
    The `requireNonNullOrElseGet' method only calls the supplier when value is needed:
    \begin{lstlisting}[language=Java]
    LocalDate hireDay = Object.requireNonNullOrElseGet(day,
        () -> new LocalDate(1970, 1, 1))
    \end{lstlisting}
\end{itemize}

\subsection{Method Reference}
\underline{Def:} \emph{method reference} directs the compiler to produce an instance of a functional interface
, overriding the \underline{single abstract} method of the interface to call the given method. \newline
e.g.
\begin{lstlisting}[language=Java]
var timer = new timer(1000, event -> System.out.println(event));
\end{lstlisting}
is the same as:
\begin{lstlisting}[language=Java]
var timer = new timer(1000, System.out::println);
\end{lstlisting}
here, \emph{\textbf{System.out::println}} is a `method reference'

Three variants of Method reference (\underline{target::methodName}):
\begin{itemize}
    \item \emph{object::instanceMethod}: e.g. \emph{System.out::println}
    \begin{lstlisting}[language=Java]
    x -> System.out.println(x)
    \end{lstlisting}
    \item \emph{Class::instanceMethod} + 2 variants: e.g \emph{String::compareToIgnoreCase}
    \begin{lstlisting}[language=Java]
    (x, y) -> x.compareToIgnoreCase(y);
    x -> this.equals(x);
    \end{lstlisting}
    \begin{itemize}
        \item \emph{this::instanceMethod}
        \item \emph{super::instanceMethod}
    \end{itemize}
    \item \emph{Class::staticMethod}: e.g. \emph{Math::pow}
    \begin{lstlisting}[language=Java]
    (x, y) -> Math.pow(x, y)
    \end{lstlisting}
\end{itemize}

\subsection{Constructor Reference}
Almost the same as `method reference', except name of the method is \emph{new}
e.g.
\begin{lstlisting}[language=Java]
Person[] people = stream.toArray(Person[]::new)
\end{lstlisting}

i.e. \emph{int[]::new} is the same as `\emph{x -$>$ new int[x]}'

\subsection{Variable Scope}
How to construct a lambda expression (three ingredients):
\begin{enumerate}
    \item A block of code
    \item parameters
    \item Value for the `\emph{free}' variables (i.e. val which is not parameter and not defined inside code)
\end{enumerate}
e.g.
\begin{lstlisting}[language=Java]
public static void repeatMessage (String text, int delay){
    ActionListener listener = event -> {
        System.out.println(text);
        Toolkit.getDefaultToolKit().beep();
    };
    new Timer(delay, listener).start();
}

repeatMessage(``Hello'', 1000);
\end{lstlisting}
N.B. Here,
\begin{itemize}
    \item `text'is a \textbf{free} variable.
    \item `Hello' is a \textbf{captured} variable.
\end{itemize}

Some Notice of the scope in lambda expression:
\begin{itemize}
    \item we can't mutate captured variable (increment)
    \item we can't refer to a variable that is mutated outside (for loop)
    \item captured var must be `effectively final' \newline 
    (i.e. it means never assigning a new value after it has been initialized)
    \item it has the same scope as a nested block -- can't have a para in lambda which has the same name as a local val
\end{itemize}

\subsection{Processing Lambda Expression}
The point of using lambdas is \underline{\textbf{deferred execution}} (i.e. execute it later)

Some cases for executing code later:
\begin{itemize}
    \item Running code in separate thread
    \item Running code multiple times
    \item Running code at right point in an algorithm
    \item Running code when something happen
    \item Running code when necessary
\end{itemize}

To accept the lambda, we need to pick a \underline{\textbf{functional interface}} e.g.:
\begin{lstlisting}[language=Java]
public interface IntConsumer{
    // this is a functional interface
    void accept(int val);
}

public static void repeat(int n, IntConsumer action){
    // this is the method to accept lambdas
    for (int i = 0; i < n; i++){
        action.accept(i);
    }
}

// this is how to call it
repeat(10, i -> System.out.println(``CountDown: '' + (9 - i)))
\end{lstlisting}
i.e. \emph{Ref:}
\begin{itemize}
    \item \underline{\textbf{Function Interface}} are on \textit{p.337}
    \item \underline{\textbf{Function Interface} for primitive types} are on \textit{p.338}
\end{itemize}

\subsection{More about Comparators}
Some common method in Comparator:
\begin{itemize}
    \item plain sort an array:
    \begin{lstlisting}[language=Java]
    Array.sort(people,
        Comparing(Person::getName));
    \end{lstlisting}
    \item two-level sort an array:
    \begin{lstlisting}[language=Java]
    Array.sort(people,
        Comparator.comparing(Person::getLastName)
        .thenComparing(Person::getFirstName));
    \end{lstlisting}
    \item user-defined sort:
    \begin{lstlisting}[language=Java]
    Array.sort(people,
        Comparator.comparing(Person::getName,
        (s,t) -> Integer.compare(s.length(), t.length())))
    \end{lstlisting}
    \item including \emph{null} in Key function:
    \begin{lstlisting}[language=Java]
    Array.sort(people, 
        comparing(Person::getMiddleName, 
            nullsFirst(natureOrder())))
    \end{lstlisting}
\end{itemize}
\emph{Ref: pp.339 - 340 \& Java doc `Comparator'}

\section{Inner Class}
Two reasons for construct an inner class:
\begin{itemize}
    \item hidden from other classes in the same package
    \item access the data from the scope where they are defined (its mother class)
\end{itemize}

\subsection{Use of an Inner Class to Access Object State}
An example of inner class:
\begin{lstlisting}[language=Java]
    public class TalkingClock{
        private int interval;
        private boolean beep;

        public TalkingClock(int interval, boolean beep){...}
        public void start(){...}

        public class TimePrinter implements ActionListener{
            //this is an inner class
        }
    }
\end{lstlisting}

Some Notices to this example:
\begin{itemize}
    \item \textbf{NOT} every `TalkingClock' has a `TimePrinter' instance field.
    \item inner class(`TimePrinter') has access to:
    \begin{itemize}
        \item its own data fields
        \item those of the outer object creating it
    \end{itemize}
    \item Inner class can be \textbf{private}, normal class can only be public
\end{itemize}

\subsection{Special Syntax Rules for inner classes}
Some new notations:
\begin{itemize}
    \item Outer class reference: ``\emph{OuterClass.\textbf{this}}''
    \item Inner object Constructor: ``\emph{outerObject.\textbf{new} InnerClass (constr para)}''
    \item Inner class reference: ``\emph{OuterClass.InnerClass}''
\end{itemize}

\subsection{Are Inner Classes Useful, Necessary, or Secure?}
Important Fact: inner classes are a phenomenon of the `\emph{compiler}', not the `\emph{virtual machine}'

\underline{Explaination:} If we define a regular class the same as the inner class, 
we will get the class with exact same functionalities but with some \emph{tags}\newline
e.g.
\begin{lstlisting}[language=Java]
public class innerClass.TalkingClock$TimePrinter
    implements java awt.event.ActionListener{
        final innerClass.TalkingClock this$0;
        public innerClass.TalkingClock$TimePrinter
            (innerClass.TalkingClock);
        public void actionPerformed
            (java.awt.even.ActionListener)
    }
\end{lstlisting}
`\emph{this\$0}' is produced by the compiler and run that in the virtual machine

\underline{Necessity:}\newline
Inner class is powerful than regular class because they have more \underline{\textbf{access privileges}}.

\underline{Security:}\newline
When it is a public accessiable inner class, it is possible to hack the class by adding class into the same package

\subsection{Local Inner Class}
\underline{Definition:} Class defined inside local Methods

Their accssors are never declared, and their scope is always restricted to the block where they are declared

\subsection{Accessing Variables from Outer Methods}
Local inner classes are powerful than other normal inner class: \textbf{they can access local variables}

Noticed that: those local varibales must be \emph{\textbf{effectively} final}

\emph{Ref: For details, pp.350 - 351}

\subsection{Anonymous Inner Classes}
\underline{Definition:} make only a single objecgt of this class, and don't give the class a name.
This is called `\emph{Anonymous Inner Classes}'\newline
e.g.
\begin{lstlisting}[language=Java]
public void start(int interval, boolean beep){
    var listener = new ActionListener(){
        public void actionPerformed(ActionEvent event){
            ...
        }
    };

    var timer = new timer(interval, listener);
    timer.start();
}
\end{lstlisting}

General Syntax is:\newline
\emph{\textbf{new} SuperType(construction parameters)\{inner class method and data\}}

Some notice of Anonymous inner classes:
\begin{itemize}
    \item \emph{superType} can be an \underline{interface} as well as a \underline{class}
    \item no constructors (class has no name)
    \item used for event listeners and other call backs, but better to use lambdas
\end{itemize}

\subsection{Static Inner Classes}
\underline{When to use:} want to hide one class inside another, don't want the inner class to have
reference to the outer class object.\newline
e.g. Self Defined data structure:
\begin{lstlisting}[language=Java]
class ArrayAlg{
    // self-defined data structure
    public static class pair{
        private double first;
        private double second;

        public Pair(double first, double second){
            this.first = first;
            this.second = second;
        }
        // getter method
        ... 
    }

    // static method using it
    public static Pair minmax(double[] values){
        ... 
        return new Pair(min, max)
    }
}
\end{lstlisting}

some Notices of Static inner class:
\begin{itemize}
    \item only inner class can be declared as \emph{static}
    \item the example requires \emph{static} inner classs(\underline{Pair}) because the inner class object is constructed inside a 
    \emph{static} method (\underline{minmax})
    \item static inner classes can have static fields and Methods
    \item inner classes inside an interface are automatically \emph{static} and \emph{public}
\end{itemize} 

\section{Service Loaders}
Materials for the second reading

\section{Proxies}
Materials for the second reading




\end{document}