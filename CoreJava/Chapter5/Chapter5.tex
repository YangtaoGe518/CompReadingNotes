\documentclass[12pt]{article}
\usepackage[utf8]{inputenc}
\usepackage{comment}
\usepackage{listings}
\usepackage{mathtools}

\setlength{\parindent}{0em}
\setlength{\parskip}{0.5em}

\title{Inheritance}
\author{Yangtao Ge}
\date{\today}

\begin{document}
\maketitle

\section{Classes, Superclasses and Subclasses}
These relationships are usually related to an `\textit{is-a}' relationship.
(i.e. Manager is an employee)

\subsection{Defining Subclasses}
\underline{Basic Grammar:}
\begin{lstlisting}[language=Java]
    public class Manager extends Employee{
        ... 
    }
\end{lstlisting}

Some feature of using `\textbf{extends}':
\begin{itemize}
    \item Subclasses $\xrightarrow{extends}$ Superclasses
    \item Subclasses have \emph{more} functionalities than their Superclasses
    \item Subclasses can use method and field from \textit{Superclasses} \newline
    (i.e. new avaliable = original + extended)
    \item point out the differences between original and new classes (\emph{\textbf{Factoring}})
\end{itemize}

\subsection{Overriding Methods}
\underline{Definition:} In a subclasses, it has the same \textbf{method (name + parameter)} as its superclass,
the functionalities are different from its original one. \newline
it is different from `\textit{overloading}' (same name + different parameter) \newline
e.g.
\begin{lstlisting}[language=Java]
    public double getSalary(){
        double baseSalary = super.getSalary();
        return baseSalary + bonus;
    }
\end{lstlisting}

Some notices:
\begin{itemize}
    \item we have no access to `\textit{salary}' (private field in Employee)
    \item we cannot plainly use `\textit{getSalary()}' (it means call it self -- infinit loop)
    \item `\textit{super}' here is not a reference to an object,\newline
     it is just for \underline{invoking superclass method}  
\end{itemize}

\subsection{Subclass Constructors}
The \textit{second} way of using `super' is \underline{build contructors}:
\begin{lstlisting}[language=Java]
public Manager(String name, double salary, 
                int year, int month, int day){
    super(name, salary, year, month, day);
    bonus = 0;
}
\end{lstlisting}

N.B. When \emph{super} Constructor is not used, no-argument Constructor is applied

\underline{Summary of `this' and `super':}
\begin{itemize}
    \item \textbf{this}:
    \begin{itemize}
        \item denote a reference to the implicit parameter
        \item call another constructor of the \underline{same class}
    \end{itemize}
    \item \textbf{super}:
    \begin{itemize}
        \item invoke the super class method
        \item invoke a \underline{superclass} constructor
    \end{itemize}
    \item when both used for `constructor': be careful about \textit{this class(this)} or \textit{superclass(super)}
\end{itemize}

**\emph{polymorphism} means an \underline{object variable} can refer to \textit{multiple} actual types.(\textit{Ref: p.217 \& subsection \ref{polymorphism}})

\subsection{Inheritance Hierarchies}
Inheritance can have \textbf{more than one} layer, \newline
e.g. Employee $\leftarrow$ Manager $\leftarrow$ Executive (this path is `\textit{inheritance chain}')

\textit{Ref: pp.216-217, especially Figure 5.1}

\subsection{Polymorphism} \label{polymorphism}
\emph{polymorphism} means an \underline{object variable} can refer to \textit{multiple} actual types.
(i.e. every Manager is an employee, but not every employee is a Manager)\newline
e.g.1 (\textit{Every Manager is employee})
\begin{lstlisting}[language=Java]
    Employee e;
    e = new Employee(...); //Employee Object expected
    e = new Manager(...); // Manager is a subclass of Employee
\end{lstlisting}

e.g.2 (\textit{Not every employee is a Manager})
\begin{lstlisting}[language=Java]
    Manager boss = new Manager(...);
    Employee[] stafflist = new Employee[3];
    stafflist[0] = boss;

    boss.setBonus(5000); // correct
    staff[0].setBonus(5000); //ERROR
\end{lstlisting}

\subsection{Understanding Method calls}
\underline{Understand what happends:}
\begin{enumerate}
    \item know all possible candidates for the method to be called
    \item knows the name and parameter types of the method
    \item \label{3} decide `\emph{static binding}' or `\emph{dynamic binding}'
    \begin{itemize}
        \item \textbf{static}: has modifier `private', `static', `final' or a constructor
        \item \textbf{dynamic}: depends on the actual type of the `\textbf{implicit} parameter' 
    \end{itemize}
    \item running the `static' or `dynamic':
    \begin{itemize}
        \item static: run the method call indicated by step \ref{3}.
        \item dynamic: call the verision of the method that is appropriate for the `\textit{actual}' type by `\textbf{method table}'
        (method table list all \textit{method signatures} and \textit{actual methods to be called})\newline
        e.g. `\emph{e.getSalary()}'
        \begin{enumerate}
            \item fetches the method table for the actual type of `\textit{e}'
            \item lookup for defining classs of the signatures `\textit{getSalary()}'
            \item get the correct method
        \end{enumerate}
    \end{itemize}
\end{enumerate}
i.e. \textbf{Method table} is: \newline
\underline{Exp:} For actual type `Manage' the following signatures are corresponding to the catual method call.\newline
Manager:
\begin{itemize}
    \item getName() $\rightarrow$ Employee.getName()
    \item getSalary() $\rightarrow$ Employee.getSalary()
    \item getHireDay() $\rightarrow$ Employee.getHireDay()
    \item raiseSalary(double) $\rightarrow$ Employee.raiseSalary()
    \item setBonus(double) $\rightarrow$ Manager.setBonus()
\end{itemize}

\subsection{Preventing Inheritance: \emph{Final} Classes and Method}

\end{document}