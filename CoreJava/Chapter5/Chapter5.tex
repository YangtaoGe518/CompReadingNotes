\documentclass[12pt]{article}
\usepackage[utf8]{inputenc}
\usepackage{comment}
\usepackage{listings}
\usepackage{mathtools}

\setlength{\parindent}{0em}
\setlength{\parskip}{0.5em}

\title{Inheritance}
\author{Yangtao Ge}
\date{\today}

\begin{document}
\maketitle

\section{Classes, Superclasses and Subclasses}
These relationships are usually related to an `\textit{is-a}' relationship.
(i.e. Manager is an employee)

\subsection{Defining Subclasses}
\underline{Basic Grammar:}
\begin{lstlisting}[language=Java]
    public class Manager extends Employee{
        ... 
    }
\end{lstlisting}

Some feature of using `\textbf{extends}':
\begin{itemize}
    \item Subclasses $\xrightarrow{extends}$ Superclasses
    \item Subclasses have \emph{more} functionalities than their Superclasses
    \item Subclasses can use method and field from \textit{Superclasses} \newline
    (i.e. new avaliable = original + extended)
    \item point out the differences between original and new classes (\emph{\textbf{Factoring}})
\end{itemize}

\subsection{Overriding Methods}
\underline{Definition:} In a subclasses, it has the same \textbf{method (name + parameter)} as its superclass,
the functionalities are different from its original one. \newline
it is different from `\textit{overloading}' (same name + different parameter) \newline
e.g.
\begin{lstlisting}[language=Java]
    public double getSalary(){
        double baseSalary = super.getSalary();
        return baseSalary + bonus;
    }
\end{lstlisting}

Some notices:
\begin{itemize}
    \item we have no access to `\textit{salary}' (private field in Employee)
    \item we cannot plainly use `\textit{getSalary()}' (it means call it self -- infinit loop)
    \item `\textit{super}' here is not a reference to an object,\newline
     it is just for \underline{invoking superclass method}  
\end{itemize}

\subsection{Subclass Constructors}
The \textit{second} way of using `super' is \underline{build contructors}:
\begin{lstlisting}[language=Java]
public Manager(String name, double salary, 
                int year, int month, int day){
    super(name, salary, year, month, day);
    bonus = 0;
}
\end{lstlisting}

N.B. When \emph{super} Constructor is not used, no-argument Constructor is applied

\underline{Summary of `this' and `super':}
\begin{itemize}
    \item \textbf{this}:
    \begin{itemize}
        \item denote a reference to the implicit parameter
        \item call another constructor of the \underline{same class}
    \end{itemize}
    \item \textbf{super}:
    \begin{itemize}
        \item invoke the super class method
        \item invoke a \underline{superclass} constructor
    \end{itemize}
    \item when both used for `constructor': be careful about \textit{this class(this)} or \textit{superclass(super)}
\end{itemize}

**\emph{polymorphism} means an \underline{object variable} can refer to \textit{multiple} actual types.(\textit{Ref: p.217 \& subsection \ref{polymorphism}})

\subsection{Inheritance Hierarchies}
Inheritance can have \textbf{more than one} layer, \newline
e.g. Employee $\leftarrow$ Manager $\leftarrow$ Executive (this path is `\textit{inheritance chain}')

\textit{Ref: pp.216-217, especially Figure 5.1}

\subsection{Polymorphism} \label{polymorphism}
\emph{polymorphism} means an \underline{object variable} can refer to \textit{multiple} actual types.
(i.e. every Manager is an employee, but not every employee is a Manager)\newline
e.g.1 (\textit{Every Manager is employee})
\begin{lstlisting}[language=Java]
    Employee e;
    e = new Employee(...); //Employee Object expected
    e = new Manager(...); // Manager is a subclass of Employee
\end{lstlisting}

e.g.2 (\textit{Not every employee is a Manager})
\begin{lstlisting}[language=Java]
    Manager boss = new Manager(...);
    Employee[] stafflist = new Employee[3];
    stafflist[0] = boss;

    boss.setBonus(5000); // correct
    staff[0].setBonus(5000); //ERROR
\end{lstlisting}

\subsection{Understanding Method calls}
\underline{Understand what happends:}
\begin{enumerate}
    \item know all possible candidates for the method to be called
    \item knows the name and parameter types of the method
    \item \label{decision} decide `\emph{static binding}' or `\emph{dynamic binding}'
    \begin{itemize}
        \item \textbf{static}: has modifier `private', `static', `final' or a constructor
        \item \textbf{dynamic}: depends on the actual type of the `\textbf{implicit} parameter' 
    \end{itemize}
    \item running the `static' or `dynamic':
    \begin{itemize}
        \item static: run the method call indicated by step \ref{decision}.
        \item dynamic: call the verision of the method that is appropriate for the `\textit{actual}' type by `\textbf{method table}'
        (method table list all \textit{method signatures} and \textit{actual methods to be called})\newline
        e.g. `\emph{e.getSalary()}'
        \begin{enumerate}
            \item fetches the method table for the actual type of `\textit{e}'
            \item lookup for defining classs of the signatures `\textit{getSalary()}'
            \item get the correct method
        \end{enumerate}
    \end{itemize}
\end{enumerate}
i.e. \textbf{Method table} is: \newline
\underline{Exp:} For actual type `Manage' the following signatures are corresponding to the catual method call.\newline
Manager:
\begin{itemize}
    \item getName() $\rightarrow$ Employee.getName()
    \item getSalary() $\rightarrow$ Employee.getSalary()
    \item getHireDay() $\rightarrow$ Employee.getHireDay()
    \item raiseSalary(double) $\rightarrow$ Employee.raiseSalary()
    \item setBonus(double) $\rightarrow$ Manager.setBonus()
\end{itemize}

\subsection{Preventing Inheritance: \emph{Final} Classes and Method}
\underline{Definition:} Classes that cannot be \underline{extended} are called \textit{final} classes. 
(i.e. Preventing others from forming a `subclass' of one of your classes)\newline
e.g.
\begin{lstlisting}[language=Java]
    public final class Executive extends Manager{
        ...
    }
\end{lstlisting}

\underline{Definition:} Method that cannot be \underline{overriden} are called \textit{final} methods.

All methods in a `\textit{final} class' are automatically \textit{final}

The \textbf{Only} good reason to make classes/methods \emph{final} is:\newline
\underline{To make its semantics cannot be changed in a subclass}

\underline{Definition:} If a method is not overriden, and it is short,
then a compiler can optimize the method call. It is called \emph{inlining} (\textit{Ref: pp.222-223 \& COMP0012 Compiler})

\subsection{Casting}
\underline{Definition:} The process of \textbf{forcing} a conversion from one type to another.\newline
e.g.
\begin{lstlisting}[language=Java]
    double x = x = 3.406;
    int nx = (int) x;
\end{lstlisting}

Some features to consider about:
\begin{itemize}
    \item Cast only within an \underline{inheritance hierarchy}
    \item Use `\emph{instanceof}' to check before casting from a superclass to a subclass\newline
    e.g.
    \begin{lstlisting}[language=Java]
    if (staff[1] instanceof Manager){
        boss = (Manager) staff[1];
    }
    \end{lstlisting}
\end{itemize}

The reason for doing a cast is \underline{use a object in its full capacity} (i.e. use special method `\emph{setBouns()}')

\subsection{Abstract Classes}
\underline{Definition:} \emph{Abstract} (class): is a superclass that cannot be instantiated 
and is used to state or define general characteristics

Some features to remember:
\begin{itemize}
    \item using `\emph{abstract}' $\rightarrow$ don't need to implement the method at all\newline
    e.g.
    \begin{lstlisting}[language=Java]
    public abstract class Person{
        private String name;
        public Person(String name){
            this.name = name;
        }
        // just a signature
        public abstract String getDescription();
        
        public String getName(){
            return name;
        }
    }
    \end{lstlisting}

    \item when extending a abstract class -- 2 choices:
    \begin{itemize}
        \item left methods undefined $\rightarrow$ tag the subclass `\emph{abstract}'
        \item define \underline{all} method $\rightarrow$ it is a normal subclass(concrete one)
    \end{itemize}

    \item Class can be tagged as `\emph{abstract}' with no abstract methods
    \item Abstract class \textbf{cannot} be instantiated $\rightarrow$ no objects can be created
    \item Abstract class \textbf{can} be \emph{object variables}, but need to refer to an object of \textit{concrete subclass}\newline
    e.g.
    \begin{lstlisting}[language=Java]
    Person p = new Student(`Yangtao.G',`Comp Sci');
    \end{lstlisting}
\end{itemize}

\textit{Ref: pp.227-229 \& Chapter 6 Interfaces}

\subsection{Protected Access}
When to use Protected:
\begin{itemize}
    \item field: two cases
    \begin{itemize}
        \item restrict a method to subclasses only
        \item allow subclass methods to access a superclass (less common)
    \end{itemize}
    N.B. Protected field is accessible by any class \underline{in the same package}, so be cautious when using item
    \item method: the subclasses can be trusted to use the method correctly.(more common to use)
\end{itemize}

\underline{\textbf{Summary of four access modifiers:}}
\begin{itemize}
    \item private: in Class only
    \item public: by the world, everywhere
    \item protect: in the package and all subclasses
    \item `no modifier': default is accessible in the package 
\end{itemize}

\section{\emph{Object:} The Cosmic Superclass}
Object method: Only cover \underline{equals, hashcode, toString} Methods, In Chapter 12 concurrency, more method will be covered

\subsection{Variables of Type `object'}
Variables `\emph{object}' can refer to \underline{any type} e.g.
\begin{lstlisting}[language=Java]
    Object obj = new Employee(``Harry Porter'', 35000);
\end{lstlisting}

Only value of \emph{\underline{primitive types}} are not objects

All array types are classes extended from `Object' class

\subsection{The `\emph{equals}' Method}
\underline{Definition:} \emph{`equals'} method tests whether one \textbf{object} is equal to another.\newline
Differents between \emph{equals} (equal) and \emph{==} (identical):
\begin{itemize}
    \item \textbf{==}: checks if both objects point to the same \underline{memory location}
    \item \textbf{equals()}: evaluates to the comparision of \underline{value in the object}
\end{itemize}
N.B. identical $\rightarrow$ equal (True); equal $\rightarrow$ identical (False)

This method usually needs implementing to do state-based equality testing (whether they are in the same state)\newline
e.g.
\begin{lstlisting}[language=Java]
public class Employee{
    ... 
    public boolean equals(Object otherObject){
        // whether the objects are identical
        if (this == otherObject) return true;
        // explicit para is null
        if (otherObject == null) return false;
        // class don't match
        if (getClass() != otherObject.getClass){
            return false;
        }
        // After checking 3 conditions above
        // it must be a non-null Employee
        Employee other = (Employee) otherObject;
        // whether the fields have identical values
        return name.equal(other.name)
            && salary == other.salary
            && Objects.equals(hireDay, other.hireDay);
    }
}
\end{lstlisting}

When defining `equals' method for a subclass, we need to call `equals' on its \emph{`superclass'}\newline
e.g.
\begin{lstlisting}[language=Java]
public class Manager extends Employee{
    ... 
    public boolean equals(Object otherObject){
        //check belongs to the same class
        if(!super.equals(otherObject)) return false;
        Manager other = (Manager) otherObject;
        return bonus == other.bonus;
    }
}
\end{lstlisting}

\subsection{Equality Testing and Inheritance}
The requirement for `\emph{equals} method':
\begin{itemize}
    \item \textbf{Reflexive}: for non-null reference x, \emph{x.equals(x)} return \underline{true}
    \item \textbf{Symmetric}: for ref x \& y, \emph{x.equals(y) $\Leftrightarrow$ y.equals(x)}
    \item \textbf{Transitive}: for ref x, y \& z, \emph{x.equals(y) $\land$ y.equals(z) $\rightarrow$ x.equals(z)}
    \item \textbf{consistent}: x, y not change $\rightarrow$ \emph{x.equals(y)} remains the same
    \item for any non-null ref x, \emph{x.equal(null)} is \underline{false}
\end{itemize}

Recipe for writting the perfect `\emph{equals} method':
\begin{enumerate}
    \item name the explicit parameter `\emph{otherObject}' (cast it to `\emph{other}' later)
    \item test itdentity:
    \begin{lstlisting}[language=Java]
    if (this == otherObject) return true;
    \end{lstlisting}
    \item test whether it is null:
    \begin{lstlisting}[language=Java]
    if (otherObject == null) return false;
    \end{lstlisting}
    \item Compare the classes of \emph{this} and \emph{otherObject}:
    \begin{itemize}
        \item If the semantics of `equals' can \underline{change in subclasses}: Using \textbf{getClass()}
    \begin{lstlisting}[language=Java]
    if(getClass() != otherObject.getClass()) return false;
    \end{lstlisting}
        \item If the \underline{same} semantics holds for \underline{all subclasses}: Using \textbf{instanceof}
    \begin{lstlisting}[language=Java]
    if(!(otherObject instanceof ClassName)) return false;
    \end{lstlisting}
    \end{itemize}
    \item Cast \emph{otherObject} to a variabe of current class type:
    \begin{lstlisting}[language=Java]
    ClassName other = (ClassName) otherObject;
    \end{lstlisting}
    \item Compare the fields:
    \begin{itemize}
        \item Use `==' for primitive type fields
        \item Use `Object.equals' for object fields
    \end{itemize}
    \begin{lstlisting}[language=Java]
    return field1 == othe.field1
        && Objecs.equals(field2, other.field2)
    \end{lstlisting}
\end{enumerate}

Common mistake: \underline{Cannot declares the method para as `explicit para'} (e.g. Employee).
protect from it by adding `\textbf{@Override}'

\subsection{The \emph{hashCode} Method}
\underline{Definition}: A hash code is an \underline{integer} that derived from an object.

Two types of \emph{hashcode} method:
\begin{itemize}
    \item default hashcode: derives the hash code from the object's \underline{memory address}
    \item redefined hashcode: has it own algorithm. e.g. Strings derives from its \emph{content}
\end{itemize}

Usually, `equals' method is related to `hashcode' method (redefine together):
\begin{itemize}
    \item \emph{x.equal(y) $\rightarrow$ x.hashCode() == y.hashCode()}
    \item define equal on employee ID $\rightarrow$ hashcode method only hashes employee ID
\end{itemize}

An examples of `hashcode' method:\newline
e.g.1
\begin{lstlisting}[language=Java]
public int hashCode(){
    return 7 * Objects.hashCode(name)
        + 11 * Double.hashCode(salary)
        + 13 * Objects.hashCode(hireDay);
}
\end{lstlisting} 

e.g.2 use `\emph{hash()} method'
\begin{lstlisting}[language=Java]
public int hashCode(){
    return Objects.hash(name, salary, hireDay);
}
\end{lstlisting}

\subsection{The \emph{toString} Method}
\emph{Ref:pp.241-243}

A normal \emph{toSrting} method:
\begin{lstlisting}[language=Java]
public String toString(){
    return getClass().getName()
        + ``[name=" + name
        + ``,salary='' + salary
        + ``,hireDay='' + hireDay
        + ``]'';
}
\end{lstlisting}


Every object can be printed because of `toString' method

It is useful for logging in Chapter 7.

\end{document}