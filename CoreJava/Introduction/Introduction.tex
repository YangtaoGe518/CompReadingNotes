\documentclass[12pt]{article}
\usepackage[utf8]{inputenc}
\usepackage{comment}
\usepackage{listings}

\setlength{\parindent}{0em}
\setlength{\parskip}{0.5em}

\title{Introduction to Java}
\author{Yangtao Ge}
\date{\today}

\begin{document}
\maketitle
\section{Preface}
This book includes:
\begin{itemize}
    \item Objective-oriented programming
    \item Reflection and proxies
    \item Interface and inner classes
    \item Exception handling
    \item Generic programming
    \item The collection framework
    \item The event listener model
    \item Graphical User Interface design with the Swing UI toolkit
    \item Concurrency
\end{itemize}
The other knowledges are includes in ``Volume II'' e.g. Stream API, Datebases, XML, Network programming etc.

\subsection*{Synopsis (10th-edition)}
\begin{itemize}
    \item chapter 1-3: Basic Introduction of Java (not focusing on)
    \item chapter 4: OOP idea and basic terminology (As a review, very fast reading)
    \item chapter 5: Inheritance
    \item chapter 6: Interface and lambda expression
    \item chapter 7: Exception handling
    \item chapter 8 Generic programming (new things for the second reading)
    \item chapter 9: collection (new things for the second reading)
    \item chapter 10-12: Java Swing (Chapter 10-11 in 11th-edition)
    \item chapter 13: deploy program (Interesting, not in 11th-edition) 
    \item chapter 14: Concurrency (new things for the second reading, Chapter 12 in 11th-edition)
\end{itemize}

\section{Chapter 1: An Introdcution to Java}
\begin{abstract}
This chapter notes will only focus on \textbf{Buzzwords} in Java and some related definition.
Details will be required to refer to the actual book.

These 11 buzzword is: 
\begin{itemize}
    \item Simple
    \item Object-oriented
    \item Distributed
    \item Robust
    \item Secure
    \item Architecture-Neutral
    \item Portable
    \item Interpreted
    \item High-Performance
    \item Multithreaded
    \item Dynamic
\end{itemize}
\end{abstract}

\subsection{Buzzword}
The detail definitions of these buzzword are as follows:
\begin{enumerate}
    \item[\textbf{Simple}] in C++ syntax, but be \textbf{simpler} to use 
    and \textbf{smaller} to install
    \item[\textbf{Object-oriented}] Focusing on the \textbf{data} and on the \textbf{interfaces} to that Object
    \item[\textbf{Distributed}] It has an extensive library of routines for coping with TCP/IP protocols like HTTP and FTP
    \item[\textbf{Robust}] It has \textbf{early checking} for possible problems, \textbf{Dynamic checking}
    and \textbf{eliminating situations} that are error-prone.
    \item[\textbf{Secure}] Three kinds of attacks impossible:
        \begin{itemize}
            \item Overrunning the running time stack
            \item corrupting memory outside its process space
            \item Reading and writting without permission
        \end{itemize}
    \item[\textbf{Architecture-Neutral}] Running on any Operating systems and Computer Architectures
    by using \underline{Java's virtual machine} 
    \item[\textbf{Portable}] No ``implementation-dependent'' i.e. The \textit{size} of data type are specified
    \item[\textbf{Interpreted}] The Java Interoreter can execute Java bytecodes on any machine to which the interpreter has been ported.
    \item[\textbf{High-Performance}] Java bytecodes are more than \textit{adequate}
    \item[\textbf{Multithreaded}] support ``Concurrent'' and has lots of benefits.
    \item[\textbf{Dynamic}] more Dynamic than C++. i.e. running time is strictly divided from compiling time    
\end{enumerate}

\subsection{Java applets and the internet}
Java program that works on web pages are called \emph{applets}. Running it needs a web browser
which can \textbf{exectue Java bytecode} for you.

This allow a web page:
\begin{itemize}
    \item reacts user commands
    \item changes its appearance
    \item exchange data between \underline{clinet} and \underline{server}
\end{itemize}

\end{document}
