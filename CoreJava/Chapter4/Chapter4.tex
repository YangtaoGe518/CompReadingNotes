\documentclass[12pt]{article}
\usepackage[utf8]{inputenc}
\usepackage{comment}
\usepackage{listings}
\usepackage{mathtools}

\setlength{\parindent}{0em}
\setlength{\parskip}{0.5em}

\title{Objects and Classes}
\author{Yangtao Ge}
\date{\today}

\begin{document}
\maketitle

\section{Introduction to Object-Oriented Programming}
What is Object-Oriented Programming:\newline
Programming with several \textbf{objects}, each object has a specific functionality
which exposed to its users, but \underline{a hidden implementation}

Two Ways of thinking:
\begin{itemize}
    \item Traditional: algorithms $\rightarrow$ data structures\newline
    Note: fine for small problems but \textit{cannot} handle large problems.
    \item Morden: data structures $\rightarrow$ algorithms\newline
    Note: More efficient to \textbf{store} data first then \textbf manipulate them
\end{itemize}

\subsection{Classes}
Class $\xrightarrow{Construct}$ Instance $\xleftarrow{Use}$ program

\textbf{Encapsulation} is the key of OOP:
\begin{itemize}
\item \textbf{Definition}: It is combining data and behavior in one package and 
hiding the implementation detail from the users of the object
\item \textbf{How}: methods \emph{never} directly access instance field in a class than its own
i.e. ``Black Box behaviour''
\end{itemize}

\subsection{Objects}
Three characteristics:
\begin{itemize}
    \item behaviour:  what can it do + what can be done to it
    \item state: how does the object react when use its method
    \item identity: how is the object distinguish from others
\end{itemize}

\subsection{Identifying Classes}
A Common begin of OOP design: \underline{Identify} the classes and \underline{Add} methods to sperate classes

Rule of Naming:
\begin{itemize}
    \item Class Name: Nouns $\rightarrow$ What it is 
    \item Method Names: Verbs $\rightarrow$ What can it do
\end{itemize}

\subsection{Relationships between classes}
Common Relations are:
\begin{itemize}
    \item[\textbf{dependence}] ``uses-a'' Express a relationship one class manipulates another class
    \item[\textbf{aggregation}] ``has-a'' Express a relationship specifying the whole and its parts
    \item[\textbf{inheritance}] ``is-a'' Express a relationship between a more special and a more general class
\end{itemize}

UML(Unified Modeling Language) notations aree used to expressed the relationship by diagram\newline
Ref: \underline{p.131 Core Java, COMP0004 Note}

\section{Using Predefined Classes}
\subsection{Objects and Object Variables}
A constructor is a \textbf{special method} whose purpose is to \underline{construct} and \underline{initialize} objects 

Key facts between Object Variables and Objects:
\begin{itemize}
    \item a variable called ``deadline'' with type ``Date'' is not a object but a variable
    \item object variables need to be initialized
    \item object variables doesen't contains an object, but it only \emph{refers} to an object
    \item Explicitly, an object variable to \textbf{null} to indicate that it currently refers to no object
\end{itemize}

Two ways of INIT:
\begin{itemize}
    \item \textit{deadline = new Date();} refers to newly constructed object
    \item \textit{deadline = birthdate;} refers to an existing object
\end{itemize}

\subsection{The `LocalDate' Class of the Java Library}
\textit{Ref: pp. 135-137 Core Java}
\subsection{Mututator and Accessor Methods}
Definitions:
\begin{itemize}
    \item Mutator method: method which will change its own original value and return
    \item Accessor method: method which will \textbf{not} modify its original value  
\end{itemize}



\end{document}