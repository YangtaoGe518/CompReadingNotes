\documentclass[12pt]{article}
\usepackage[utf8]{inputenc}
\usepackage{comment}
\usepackage{listings}

\setlength{\parindent}{0em}
\setlength{\parskip}{0.5em}

\title{Block Two: The Information Layer}
\author{Yangtao Ge}
\date{\today}

\begin{document}
\maketitle

\section{Chapter 2: Binary Value and Number System}
\begin{abstract}
This chapter describes binary values -- the way in which computer \textbf{hardware} represents and manages information.
It also puts the binary value in all number system.
\end{abstract}

\subsection{Number and  Computing}
Some definitions of Numbers:
\begin{itemize}
    \item Number: A unit of an abstract mathematical system subject to \underline{the laws of arithmetic} (succession, addition and multiplication).
    \item Natural number: The number \textbf{0} and any number obtained by \underline{reaptedly adding to 1} to 1.
    \item Negative number: A value less than zero and with a sign oppsite to its \textbf{positive counterpart}
    \item Rational number: An integer or the \underline{quotient} of two integers (division by zero included)
\end{itemize}

\subsection{Positional Notation}
Some definitions of Base:
\begin{itemize}
    \item Base: The foundational value of a number system, which dictates \textbf{number digits} and the \textbf{value of digit Position}
    \item Positional notation: A way of expressing number in different base system in a following way:
    \begin{equation}
        d_n * R^{n-1} + d_{n-1} * R^{n-2} + ... + d_2 * R + d_1
    \end{equation}
    where \textbf{Base-R} has \textit{n} digits and $d_i$ represents
    the digit in the \textit{i}th position
\end{itemize}

Watch out the digit in a number. e.g. 2074 does not have base \textbf{less than Base-8}
because digit 7 is used here.

\textbf{2 digits} is needed to represent the base value. e.g. 10 is \underline{ten} in decimal.
10 is \underline{eight} in base 8. 10 is \underline{two} in binary.

Carry and borrow system is also applied to other base system. However, the value represented binary
these carries and borrows means the \textbf{value of the base}.

All power of 2 number system can be transfered to \textbf{binary}, then to \textbf{decimal}.
Examples are as follows:
\begin{center}
\underline{count every 4 digits for Hex}

1010110 = 101(5) \& 0110(6)  

\underline{count every three digits for Oct}

101010111100 = 101(5) \& 010(2) \& 111(7) \& 100(4)
\end{center}

Algorithm for Base 10 to Other Bases is as follows:
\begin{lstlisting}
    WHILE (the quotient is not zero):
        Divide the decimal number by the new base
        Make the reminder the next digit to the left in the answer
        Replace the decimal number with the quotient
\end{lstlisting}    
This algorithm shows that:
\begin{itemize}
    \item The production of new number is \textbf{from right to left}    
    \item Quotient is repeatedly used, reminder is the \textbf{answer}
\end{itemize}

some definitions about bit:
\begin{itemize}
    \item binary digit: A digit in the \textbf{binary number} system
    \item bit: Binary digit
    \item byte: \textbf{Eight} binary digits
    \item word: A group of one or more \underline{bytes} \newline
    \emph{the number of bits in a word = word length of the computer}
\end{itemize}

\section{Chapter 3: Data Representation}



\end{document}