\documentclass[12pt]{article}
\usepackage[utf8]{inputenc}
\usepackage{comment}
\usepackage{listings}
\usepackage{mathtools}

\setlength{\parindent}{0em}
\setlength{\parskip}{0.5em}

\title{Block Two: The Information Layer}
\author{Yangtao Ge}
\date{\today}

\begin{document}
\maketitle

\section{Chapter 2: Binary Value and Number System}
\begin{abstract}
This chapter describes binary values -- the way in which computer \textbf{hardware} represents and manages information.
It also puts the binary value in all number system.
\end{abstract}

\subsection{Number and  Computing}
Some definitions of Numbers:
\begin{itemize}
    \item Number: A unit of an abstract mathematical system subject to \underline{the laws of arithmetic} (succession, addition and multiplication).
    \item Natural number: The number \textbf{0} and any number obtained by \underline{reaptedly adding to 1} to 1.
    \item Negative number: A value less than zero and with a sign oppsite to its \textbf{positive counterpart}
    \item Rational number: An integer or the \underline{quotient} of two integers (division by zero included)
\end{itemize}

\subsection{Positional Notation}
Some definitions of Base:
\begin{itemize}
    \item Base: The foundational value of a number system, which dictates \textbf{number digits} and the \textbf{value of digit Position}
    \item Positional notation: A way of expressing number in different base system in a following way:
    \begin{equation}
        d_n * R^{n-1} + d_{n-1} * R^{n-2} + ... + d_2 * R + d_1
    \end{equation}
    where \textbf{Base-R} has \textit{n} digits and $d_i$ represents
    the digit in the \textit{i}th position
\end{itemize}

Watch out the digit in a number. e.g. 2074 does not have base \textbf{less than Base-8}
because digit 7 is used here.

\textbf{2 digits} is needed to represent the base value. e.g. 10 is \underline{ten} in decimal.
10 is \underline{eight} in base 8. 10 is \underline{two} in binary.

Carry and borrow system is also applied to other base system. However, the value represented binary
these carries and borrows means the \textbf{value of the base}.

All power of 2 number system can be transfered to \textbf{binary}, then to \textbf{decimal}.
Examples are as follows:
\begin{center}
\underline{count every 4 digits for Hex}

1010110 = 101(5) \& 0110(6)  

\underline{count every three digits for Oct}

101010111100 = 101(5) \& 010(2) \& 111(7) \& 100(4)
\end{center}

Algorithm for Base 10 to Other Bases is as follows:
\begin{lstlisting}
    WHILE (the quotient is not zero):
        Divide the decimal number by the new base
        Make the reminder the next digit to the left in the answer
        Replace the decimal number with the quotient
\end{lstlisting}    
This algorithm shows that:
\begin{itemize}
    \item The production of new number is \textbf{from right to left}    
    \item Quotient is repeatedly used, reminder is the \textbf{answer}
\end{itemize}

some definitions about bit:
\begin{itemize}
    \item binary digit: A digit in the \textbf{binary number} system
    \item bit: Binary digit
    \item byte: \textbf{Eight} binary digits
    \item word: A group of one or more \underline{bytes} \newline
    \emph{the number of bits in a word = word length of the computer}
\end{itemize}

\section{Chapter 3: Data Representation}
\begin{abstract}
This chapter includes how to store a certain type of information and represent in a computer environment
\end{abstract}

\subsection{Data and Computers}
Some definitions related to data:
\begin{itemize}
    \item Data: basic value and facts
    \item Information: Organized data and can provide \textbf{useful solutions} to problems
    \item Multimeadia: Sevral different media types i.e. Numbers, Text, Audio, images and etc.
    \item Bandwidth: The number of bits or bytes that can be transmitted from one place to 
    another \underline{within a fixed time}
    \item Data compression: shrink the size of the data
    \item Compression ratio:
    \begin{equation}
        Ratio = \frac{Compressed\ Size}{Original\ Size}
    \end{equation}
    $0 < Ratio < 1$, closer to zero $\rightarrow$ tighter the compression
    \item Lossless: \underline{Without any Loss} in the process of compaction
    \item Lossy: \underline{Is lost} in the process of compaction
\end{itemize}

Real world is \textbf{infinite}, but computer is \textbf{finite}

Some definitions about types of data:
\begin{itemize}
    \item Analog data: A \textbf{continuous} representation of data
    e.g. mercury thermometer (\underline{smooth wave})
    \item Digital data: A \textbf{discrete} representation of data
    e.g. button (\underline{square wave})
\end{itemize}

In computer:
\begin{itemize}
    \item Analog Data $\xrightarrow{\text{digitize}}$ Digital Data 
    \item use \textbf{binary} system to represent them     
\end{itemize}

Degraded: Electronic signals degrades as they move down a line (\textbf{Threshold})

Some definitions about Digital signals:
\begin{itemize}
    \item Pulse-Code Modulation (PCM): Variation in a signal that jumps sharply between two \textbf{extremes}
    \item Reclocked: The act of reasserting an original digital signal before \textbf{too much degreadation occurs}
\end{itemize}

\underline{Analog vs Digital:} (need review)
\begin{itemize}
    \item[\textbf{Analog}] degrades $\rightarrow$ in-range value $\rightarrow$ valid $\rightarrow$ information lost
    \item[\textbf{Digital}] degrades $\rightarrow$ PCM $\rightarrow$ high to low $\rightarrow$ reclocked $\rightarrow$ information saved  
\end{itemize}

\emph{n} bits can represent $2^{n}$ things.\newline
Increase the number of bits by 1 $\Rightarrow$ \textbf{double} the number of things we can represent

\subsection{Representing Numeric Data}
\subsubsection{Negative Values}
\underline{The work flow is:}\newline
Sign-Magnitude Representation $\rightarrow$ Fixed-sized Numbers $\rightarrow$ Two's Complement

\begin{itemize}
    \item Sign-Magnitude Representation: ``value + sign'' \newline
    Problem: Will have \textbf{two} representation of 0 (+0 \& -0)
    \item Fixed-sized Numbers: use half of the integers to represent negatives \newline
    Method: Add the number together and \textbf{dicard} any carries
    \begin{equation}
        Negative(I) = 10^k - I
    \end{equation}
    Problem: Can't be represnet in computer
    \item Two's Complement: use certain number of bits to represent a integer and \underline{leftmost} one bit for representing \textbf{sign}
    e.g. -(2) is 11111110 \newline
    Method: \textbf{invert} the bits and \textbf{add 1}
    \begin{equation}
        Negative(I) = 2^k - I
    \end{equation}
\end{itemize}

\emph{Overflow} occurs when the value that we compute cannot fit into \underline{the number of bits} we have allocated for the result\newline
e.g. 01111111(127) + 00000011(3) = 10000010(-126) is not +130

\subsubsection{Real Numbers}
Different from Math: all noninteger values $\Leftrightarrow$ Real Number

\emph{Radix} means the \textbf{dot} that separates the \underline{whole} 
part from the \underline{fractional} part in a real number in \textit{any base}

\emph{Floating Point} means a representation of a real number that keeps track of the \textbf{sign}, \textbf{mantissa}, and \textbf{exponent}

Base-10:
\begin{equation}
    R = sign * mantissa * 10^{exp}
\end{equation}
Base-2:
\begin{equation}
    R = sign * mantissa * 2^{exp}
\end{equation}

Floating Point needs 64 bits: $64 = 1(sign) + 11(exponent) + 52(mantissa)$ i.e. double precision

\underline{Algorithm} Converting fractional parts from base-10 to other:
\begin{lstlisting}
    WHILE (the fractional part is not zero):
        Multiply the fractional part by the new base
        Make the whole part the next digit to the left in the answer
        Replace the fractional part with the result of multiplication
\end{lstlisting}

Noticed that:
\begin{itemize}
    \item it is possible that the loop will \textbf{never end} $\rightarrow$ precision problems
    \item instead of division, \textbf{multiplication} is used here
    \item More detail method of computing the Floating point is \textbf{NOT} included in this book
\end{itemize}

\end{document}