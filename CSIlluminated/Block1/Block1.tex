\documentclass[12pt]{article}
\usepackage[utf8]{inputenc}
\usepackage{comment}

\setlength{\parindent}{0em}
\setlength{\parskip}{0.5em}

\title{Block One: Overview}
\author{Yangtao Ge}
\date{\today}

\begin{document}
\maketitle

\section{Brief Contents}
\begin{itemize}
    \item Laying the Groundwork
    \item The Information Layer
    \item The Hardware Layer
    \item The programming Layer
    \item The Operating System Layer
    \item The Application Layer
    \item The Communicaton Layer
    \item Conclusion
\end{itemize}

\section{Preface}
\subsection{Organization}
Computer system is like an \textbf{onion}. The processor with its \underline{machine language} forms
the \textbf{heart} of the ``onion''.

This book is designed to:
\begin{itemize}
    \item provide an overview of the \textit{layers}.
    \item introducing the underlying \underline{hardware and software} technologies
    \item give an appreciation
    \item understanding of all aspects of computing
\end{itemize}

This book choose a method called: \textbf{inside-out} method. 
(i.e. begin with the concrete machine andexamine the layers in the order in which they were created)

\subsection{Synopsis}
\begin{itemize}
    \item chapter 1: explaining the rationale for \textbf{this book Organization}
    \item chapter 2-3: Information Layer (\textbf{{how data is represented in the computer}})
    \item chapter 4-5: Hardware Layer (\textbf{electronic circuitry and logical gates to \underline{CPU}})
    \item chapter 6-9: programming Layer (\textbf{from machine language to programming language})     
    \item chapter 10-11: Operating system Layer
    \item chapter 12-14: Application Layer (\textbf{information systems and AI})
    \item chapter 15-17: Communicaton Layer (\textbf{introducing how computer communicate, WWW, security problem})
    \item chapter 18: Conclusion with its limitations 
\end{itemize}

\section{Chapter 1: The big picture}
\begin{abstract}
    This chapter provides some \textbf{common terminology} and creating the basic platform for explorating CS 
\end{abstract}

\subsection{Computing Systems}
Computer is a device but computer system is a \textbf{dynamic entity}, to interact with its environment.
Computer system is composed of \underline{hardware} and \underline{software}.

Layer of a computing system is a flow from inside to outside:
\begin{center}
    \textbf{Information $\rightarrow$ Hardware $\rightarrow$ programming $\rightarrow$ Operating System $\rightarrow$ Application $\rightarrow$ Communicaton}
\end{center}

Computer do only very simple task, but finish them so bindingly \textbf{fast}
that many simple tasks can be combined to acomplish larger, more complicated tasks.

6 Layers explaining in detail:
\begin{enumerate}
    \item[\textbf{Information}] It is a \textit{conceptual} layer. This Layer is about understanding binary format and transforming other format into binary system.
    \item[\textbf{Hardware}] This Layer investigates using \textbf{electronic circuirty} to control the flow of electricity (i.e. using CPU)
    \item[\textbf{Programming}] This Layer is about the \textbf{instructions} to accomplish computations and manage data
    \item[\textbf{OS}] It is the \textbf{key} to understand computer system. This layer is to help manage the computer's resources
    \item[\textbf{Application}] This layer focuses on solving real-world problems
    \item[\textbf{Communicaton}] This layer explores how different computer communicates with each other   
\end{enumerate}

\textbf{abstraction} is a \underline{mental model} that removes complex details.
It is the \textbf{key} to computing 

\subsection{The history of computing}
In 1936, \textbf{Alan M. Turing} invented an abstract mathematical model called a \textit{Turing machine}.

This section includes the history of ``Hardware'' and ``Software'' (not very important) as follows:
\begin{itemize}
    \item Hardware
    \begin{enumerate}
        \item First Generation(1951-1959): Using \textbf{vacuum tubes}
        \item Second Generation(1959-1965): Using \textbf{transistor} 
        \item Third Generation(1965-1971): Using \textbf{circuit boards}
        \item Fourth Generation(1971-now): Using \textbf{Large-scale integration} silicon chip
    \end{enumerate}
    \item Software
    \begin{enumerate}
        \item First Generation(1951-1959): Written in \textbf{machine language} and develop a tool called \textbf{Assembly language}
        \item Second Generation(1959-1965): Written in \textbf{high-level languages} \underline{without} Operating system
        \item Third Generation(1965-1971): Focusing on \textbf{Operating systems} and using some Application packages
        \item Fourth Generation(1971-1989): Written in \textbf{structured programming} e.g. C++
        \item Fifith Generation(1989-now): Developing in \textbf{object-oriented} and \textbf{web design}
    \end{enumerate}
\end{itemize}

\subsection{Computing as a tool and a discipline}
\underline{Tool}: everyone excpet for toolmakers.

\underline{Discipline}: focusing on a field of study

\end{document}
