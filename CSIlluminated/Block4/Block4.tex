\documentclass[12pt]{article}
\usepackage[utf8]{inputenc}
\usepackage{comment}
\usepackage{listings}
\usepackage{mathtools}

\setlength{\parindent}{0em}
\setlength{\parskip}{0.5em}

\title{Block 4: The Programming Layer}
\author{Yangtao Ge}
\date{\today}

\begin{document}
\maketitle

\section{Chapter 6: Low-Level Programming Languages and pseudocode}
\begin{abstract}
    This Chapter focuses on how to use a `computer system'. The processing of interpreting is a from
    lowest to more advanced programming language.\newline
    i.e. machine code $\rightarrow$ assembly language $\rightarrow$ pseudocode\newline
    * This Chapter is related to ENGF0001(1) 
\end{abstract}

\subsection{Computer Operations}
\underline{Definition:} Computer is a programmable electronic device which can \textit{store, retrieve} and \textit{process} data

Four Operations:
\begin{itemize}
    \item programmable: manipulate data are stored within the machine along with the data
    \item store: store data into the memory of the machine
    \item retrieve: retrieve data from memory of the machine
    \item process: process data in some way in the arithmetic/logic unit(ALU)
\end{itemize}

\subsection{Machine Language}
Just understanding some basic knowlegde about Machine Language is fine

Each type of computer(i.e. CPU) has limited number of machine language to be executed.

\emph{Virtual Computer machine} is A hypothetical machine designed to \textbf{illustrate} important \underline{features} of a real machine
(i.e. it is not real, just a intimation)

Following subsection uses Pep/8 as a example

\subsubsection{Features of Pep/8}
65536 bytes in total (i.e. 1 byte = 8 bits)\newline
Word length is 2 bytes\newline
ALU is 16 bits\newline

Three components:
\begin{itemize}
    \item Program counter(PC): Contains the address of thenext instruction to be executed
    \item Instruction register(IR): Contains a copy pf the instruction being executed
    \item Accumulator: a register
\end{itemize}

address is a \underline{physical representation} of the memory (i.e. address is a \emph{name} of memory),
\textbf{it is not the actual place for storing data}

The address range is: 0000 -- FFFF (in Hex Decimal)

\subsubsection{Instruction Format}
Picture \emph{Ref: pp.155-156}

The Two parts of an Instruction:
\begin{itemize}
    \item Instruction specifier: 8 bits
    \item Operand specifier: 16 bits
\end{itemize}

The \underline{instruction specifier} part of an instruction:
\begin{itemize}
    \item Operation code(1-4/1-5): specify which register to use
    \item Register specifier(5): 0 for register A(Accumulator)
    \item Addressing mode(6-8): how to interpret the Operand part of the instruction.
\end{itemize}

Two mode of addressing:
\begin{itemize}
    \item Immediate: Operand is in the operand specifier of \underline{the instruction}
    \item Direct: Operand is \underline{the memory address} named in the operand specifier
\end{itemize}

\subsubsection{Example Instruction}
Instruction specifier:\newline
1 1 0 0 0 0 0 0 \newline
Operand specifier:\newline
0 0 0 0 0 0 0 0 0 0 0 0 1 1 1

\underline{Meaning:} under `Immediate Mode', Load the operand into the A register.

\subsubsection{Instruction table}
\begin{tabular}{|p{3cm}||p{10cm}|}
    \hline
    Opcode & Meaning of Instruction\\
    \hline
    \hline
    0000 & Stop execution\\
    1100 & Load operand into the A register\\
    1110 & Store the contents of the A register into the operand\\
    0111 & Add the operand to the A register\\
    1000 & Substract the operand to the A register\\
    01001 & Character input to the operand\\
    01010 & Character output from the operand\\
    \hline
\end{tabular}

\subsection{A Program Example}
\textit{Ref: pp.160-165}
This section detially metioned about the running process of machine language

\subsection{Assembly Language}
\underline{Definition:}
\begin{itemize}
    \item Assembly language: A low-level programming language in which a mnemonic represents each of the
    machine-language instruction for a particular computer
    \item Assembler: A program that \emph{translate} an assembly-language program in machine code
    \item Assembler directive: \emph{Instructions} to the translating program
\end{itemize}

N.B. Different Machine has different assembly language.

\underline{Assembly process}\newline
Program in assembly language $\xrightarrow{input}$ Assembler $\xrightarrow{output}$ Program in machine code

**\emph{Ref: pp.167-174 A lot of references of real code, also refer to ENGF0001(1) coursework}

\subsection{Expressing Algorithms}
Using \underline{\textit{Pseudocode}}

\emph{Pseudocode} is a language that follows us to express Algorithms in a clearer form.

\subsubsection{Pseudocode Functionality}
Grammar rules:
\begin{itemize}
    \item Variables: reflect the content in the Algorithms
    \item Assignment: `Set sum to 0' == `sum $\leftarrow$ 1'
    \item Input: `Read num'
    \item Output: `Write \textit{The number of value to read and sum}
    \item Selection(if-else):
    \begin{lstlisting}
        IF(sum < 0)
            Print error message
        ELSE
            Print sum
    \end{lstlisting}
    \item Repitition(while):
    \begin{lstlisting}
        Set Limit to number of values to sum
        WHILE (counter < limit)
            Read num
            sum <- sum + num
            counter <- counter + 1
    \end{lstlisting}
\end{itemize}

\subsubsection{Writing a Pseudocode Algorithm}
\textit{Ref: pp.181 -- 185}
\begin{lstlisting}
Write ``How many pairs of value are to be entered?''
Read numOfPairs
pairsRead <- 0
WHILE (numberRead < numOfPairs)
    Write ``Enter two values separated by a blank; return''
    Read num1
    Read num2
    IF (num1 < num2)
        Print num1, `` '', num2
    ELSE
        Print num2, `` '', num1
    numberRead <- numberRead + 1
\end{lstlisting}




\end{document}
