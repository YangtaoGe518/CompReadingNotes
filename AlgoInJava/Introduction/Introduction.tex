\documentclass[12pt]{article}
\usepackage[utf8]{inputenc}
\usepackage{comment}
\usepackage{listings}
\usepackage{mathtools}
\usepackage{hyperref}


\setlength{\parindent}{0em}
\setlength{\parskip}{0.5em}

\title{Introduction}
\author{Yangtao Ge}
\date{\today}

\begin{document}
\maketitle

\section{Preface}
\subsection{Purpose}
\underline{How does the books go}: \newline
\textit{Specific problems = Coding + Math Analysing}

\underline{Knowlegde preferred:}
\begin{itemize}
    \item intermediate programming(OOP \& recursion)
    \item discrete Math -- \emph{Ref: COMP0147 \& ``Discrete Mathematics and Its Application''}
\end{itemize}

\subsection{Overview}
\boxed{Part 1: Basic\  Knowlegde}
\begin{itemize}
    \item Chapter 1: Reviewing material on discrete math \& recursion + Java related(out of date, not focus on)
    \item Chapter 2: Algorithm analysis (important and doing exercise)
    \item Chapter 3: List, Stack and Queues
    \item Chapter 4: Tress (Basic, AVL \& game trees refer to advanced part)
    \item Chapter 5: Hash tables 
    \item Chapter 6: Priority Queues
    \item Chapter 7: Sorting
    \item Chapter 8: Disjoint set
    \item Chapter 9: Graph Algorithm
\end{itemize}

\boxed{Part2: Advanced\ Knowlegde}
\begin{itemize}
    \item Chapter 10: Algorithm on problem-solving techniques (Lots of Examples)
    \item Chapter 11: amortized analysis(Three data structure from C4 \& C6 + Fibonacci heap)
    \item Chapter 12: Search tree Algorithms(advanced trees)
\end{itemize}

\subsection{Exercise}
From easy to hard(marked with *), Last question demo the whole Chapter
\textit{Ref: \url{www.pearsonhighered.com/cssupport}}

\section{Chapter 1: Introduction}
\subsection{What is the Book About?}
\underline{Running code fast and analysis them}

N.B. detail contents for every chapter are in the previous section

\subsection{Mathematics Review}
\emph{Ref: pp.3-8}


\end{document}