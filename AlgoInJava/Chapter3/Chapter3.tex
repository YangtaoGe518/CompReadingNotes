\documentclass[12pt]{article}
\usepackage[utf8]{inputenc}
\usepackage{comment}
\usepackage{listings}
\usepackage{mathtools}
\usepackage{hyperref}
\usepackage{amsthm}

\theoremstyle{definition}
\newtheorem{definition}{Definition}[section]

\newtheorem{theorem}{Theorem}[section]

\DeclarePairedDelimiter \abs{\lvert}{\rvert} % short cut for absolute value

\setlength{\parindent}{0em}
\setlength{\parskip}{0.5em}

\title{Chapter 3: Lisr, Stacks, and Queues}
\author{Yangtao Ge}
\date{\today}

\begin{document}
\maketitle
\begin{abstract}
This section discusses about:
\begin{itemize}
    \item Introduce Abstratc Data Type (ADT)
    \item how to efficiently perform operations on lists 
    \item stack ADT
    \item queue ADT
\end{itemize}
\end{abstract}

\section{Abstract Data Types (ADT)}
\underline{Definition:} \emph{Abstract data type} is a set of \underline{objects} together with
a set of \underline{operations}

The three data structures (Lists, Stacks, and Queues) are ADT examples

\section{The List ADT}
Some Feature of List:
\begin{itemize}
    \item general form is: $A_0, A_1,A_2, ..., A_{N-1}$
    \item special list of size 0 is \textbf{empty list}
    \item $A_i$ succeeds $A_{i-1}$ \& $A_{i-1}$ precedes $A_i$
    \item First element is $A_0$ \& Last element is $A_{N-1}$
    \item position  of element $A_i$ is $i$
\end{itemize}
\subsection{Simple Array Implementation of Lists}
Using plain array: Only use array accesses (i.e. \emph{findkth} operation)

\textit{Ref: pp.58 - 59}

\subsection{simple Linked Lists}
feature of linked list:
\begin{itemize}
    \item consists of a series of nodes 
    \item each node contains the `\emph{element}' and `\emph{next} link'
    \item the last cell's `last link' is \emph{null}
\end{itemize}

Some avaliable method definition:
\begin{itemize}
    \item findkth: scan through the list and find the element on that position
    \item find: find the position that the element we specify 
    \item remove: method can be executed in one \emph{next} reference change
    \item insert: requires obtaining a new node from the system by using a \emph{new} call
    and executing two reference maneuvers
\end{itemize}

When removing the last element: \underline{tricky} $\rightarrow$ using double linked list

\section{List in the Java Collection API}
\subsection{\emph{Collection} Interface}
Some feature of \emph{Collection} Interface:
\begin{itemize}
    \item package in \emph{java.util}
    \item collection extends the \emph{Iterable} Interface
    \item can use enhance for loop
    \item method in Collection Interface:
\begin{lstlisting}[language=Java]
public interface Collection<AnyType> extends Iterable<AnyType>{
    int size();
    boolean isEmpty();
    void clear();
    boolean contains(AnyType x);
    boolean add(AnyType x);
    boolean remove(AnyType x);
    java.util.Iterator<AnyType> iterator();
}
\end{lstlisting}
\end{itemize}




\end{document}